\chapter{Einführung und Projektidee}
Im Rahmen der Vorlesung \glqq Media Interactive\grqq wurden wir vor die Aufgabe gestellt, ein interaktives, nicht-lineares Video zu erstellen. Das Thema des Videos konnten wir frei nach unseren Interessen wählen.Wir entschieden uns dazu, ein Video über den Brühl Boulevard, auf dem wir beide wohnen, zu erstellen.
\\ \\
Der Brühl war zur Zeit der Deutschen Demokratischen Republik und auch nach der Wende ein beliebtes Wohn- und Gewerbeviertel. Mit zahlreichen Geschäften, vielen Arbeitsplätzen sowie für die damalige Zeit modernen Wohnungen zog die Flaniermeile Menschen von Nah und Fern an. Leider verlor der Brühl nach der Wende durch die Konzentration des Einzelhandels in Einkaufszentren am Stadtrand sowie einer verfehlten Stadtentwicklungsstrategie maßgeblich an Attraktivität, was in Verbindung mit dem demografischen Wandel zu einem enormen Leerstand führte. Seit einigen Jahren jedoch plant die Stadt eine Reaktivierung des Brühls als innerstädtisches Wohngebiet für alle Generationen und Zielgruppen. Durch das gleichzeitig hohe private Engagement der Anwohner für eine Belebung ihres Boulevard sind erste Erfolge bereits deutlich sichtbar.
\\ \\
Eine Möglichkeit Menschen den Brühl noch weiter näherzubringen wäre eine virtuelle, videobasierte Tour über den Boulevard. Angelehnt an Google's bildbasierten Street View Dienst, könnte man unsere videobasierte Tour über den Brühl weitaus umfangreicher umsetzen: 360 Grad Videoaufnahmen, gegebenenfalls in Kombination mit Virtueller Realität und 3D-Technik könnten die Immersion in die virtuelle Umgebung erhöhen. Weiterhin sind Aufnahmen in Abhängigkeit von der Tags- und Jahreszeit oder tagesaktuelle Aufnahmen, zum Beispiel von Veranstaltungen, denkbar. Man müsste sich auch nicht auf den Boulevard an sich beschränken, sondern könnte auch Innenaufnahmen von Geschäften und Läden oder Luftaufnahmen, zum Beispiel die einer Drohne, in das Projekt integrieren. Interessant wäre zudem ein direkter Vergleich aktueller Aufnahmen mit historischen Bild- und Videoaufnahmen aus der identischen Aufnahmeposition und der gleichen Perspektive.
\\ \\
Eine vielfältige Endgeräteunterstützung würde den Zugang zum Projekt erleichtern. Neben den klassischen Geräten wie PC, Smartphone und Tablet sind auch moderne Formen der multimedialen Endgeräte denkbar, wie zum Beispiel VR-Headsets oder Google Glass.



\chapter{Design und Konzeption}
Im Fokus unseres Players steht das Wiedergabefenster mit dem entsprechend gerade abgespielten Video. Darunter befindet drei Boxen, die Informationen zum Video bereitstellen und/oder eine Interaktion zulassen. In der ersten Box auf der linken Seite wird dem Benutzer die Blickrichtung sowie der Ort der Aufnahme in Form der Straßenbezeichnung und der geographischen Koordinaten angezeigt. Die mittlere Box ermöglicht das navigieren im interaktiven Video. Auf diese Weise kann sich der Benutzer virtuell über den Brühl Boulevard bewegen. Falls zum gerade angezeigten Video ein Point-of-Interest zur Verfügung steht, wird auf der rechten Seite eine weitere Box eingeblendet. In dieser wird die entsprechende Bezeichnung und Beschreibung des POI's sowie ein Link zu einer Website mit weiterführenden Informationen angezeigt. Weiterhin wird bei einem vorhandenen Point-of-Interest zudem ein Overlay mit der Bezeichnung des POI eingeblendet.\\ \\
Statt die Navigationspfeile in einer Box unter dem Wiedergabefenster unterzubringen, wäre es auch denkbar gewesen, diese als Overlay über dem Video anzuzeigen. \\
Weiterhin wäre es möglich gewesen, anhand der zu jedem Video verfügbaren geographischen Koordinaten eine interaktive Landkarte im Player einzublenden, die sowohl die aktuelle Position als auch alle anderen Aufnahmepositionen visualisiert sowie durch einfaches Klicken zusätzlich die Möglichkeit der Navigation zu diesen ermöglicht hätte.



\chapter{Ablaufplan}
In der Tabelle \ref{videoübersicht} auf der folgenden Seite sind die insgesamt 26 im Projekt verwendeten Videos aufgelistet. Zu jedem Video ist die Video-ID, der Aufnahmestandort, die Blickrichtung in Form der bekannten Himmelsrichtungen, die mit dem Video verknüften Nachbarvideos sowie, falls vorhanden, der Point-of-Interest angegeben.
\begin{table}[]
\centering
\caption{Videoübersicht}
\label{videoübersicht}
\begin{tabular}{lllll}
ID & Position                                                                    & Blickrichtung & POI                                                                              & Nachbarn                                                     \\
2  & Brühl/Georgstraße                                                           & WNW           & -                                                                         & 3, 4                                                         \\
3  & Brühl/Georgstraße                                                           & SSW           & -                                                                                & 2, 4                                                         \\
4  & Brühl/Georgstraße                                                           & NNO           & Brühltor                                                                                & 2, 3, 8                                                      \\
5  & \begin{tabular}[c]{@{}l@{}}Brühl/Untere\\ Aktienstraße\end{tabular}         & WNW           & -                                                                                & 6, 7, 8                                                      \\
6  & \begin{tabular}[c]{@{}l@{}}Brühl/Untere\\ Aktienstraße\end{tabular}         & SSW           & -                                                                                & 3, 5, 7, 8                                                   \\
7  & \begin{tabular}[c]{@{}l@{}}Brühl/Untere\\ Aktienstraße\end{tabular}         & OSO           & -                                                                                & 5, 6, 8                                                      \\
8  & \begin{tabular}[c]{@{}l@{}}Brühl/Untere\\ Aktienstraße\end{tabular}         & NNO           & -                                                                                & 5, 6, 7, 9                                                   \\
9  & Brühl/Hermannstraße                                                         & NNO           & -                                                                                & \begin{tabular}[c]{@{}l@{}}10, 11, 12,\\ 13, 30\end{tabular} \\
10 & Brühl/Hermannstraße                                                         & OSO           & -                                                                                & 9, 11, 12                                                    \\
11 & Brühl/Hermannstraße                                                         & WNW           & -                                                                                & 9, 10, 12                                                    \\
12 & Brühl/Hermannstraße                                                         & SSW           & -                                                                                & 6, 9, 11                                                     \\
13 & Brühl                                                                       & NNO           & -                                                                                & 15, 16, 17                                                   \\
15 & Brühl                                                                       & SSW           & -                                                                                & 12, 14, 16                                                   \\
16 & Brühl                                                                       & WNW           & \begin{tabular}[c]{@{}l@{}}Rosa-Luxemburg-\\ Grundschule\\ am Brühl\end{tabular} & 14, 15                                                       \\
17 & Brühl/Elisenstraße                                                          & NNO           & Schriftzug \glqq Zuhause\grqq                                                             & 18, 19, 20, 21                                               \\
18 & Brühl/Elisenstraße                                                          & SSW           & -                                                                                & 15, 17, 19, 20                                               \\
19 & Brühl/Elisenstraße                                                          & OSO           & -                                                                                & 18, 20, 21                                                   \\
20 & Brühl/Elisenstraße                                                          & WNW           & -                                                                                & 18, 19, 21                                                   \\
21 & Brühl/Elisenstraße                                                          & NNO           & -                                                                                & 19, 20, 22, 27                                               \\
22 & Brühl/Elisenstraße                                                          & SSW           & -                                                                                & 18, 19, 20, 21                                               \\
24 & \begin{tabular}[c]{@{}l@{}}Brühl/Zöllnerstraße/\\ Zöllnerplatz\end{tabular} & SSW           & -                                                                                & 22, 25, 26, 27                                               \\
25 & \begin{tabular}[c]{@{}l@{}}Brühl/Zöllnerstraße/\\ Zöllnerplatz\end{tabular} & OSO           & -                                                                                & 24, 26, 27                                                   \\
26 & \begin{tabular}[c]{@{}l@{}}Brühl/Zöllnerstraße/\\ Zöllnerplatz\end{tabular} & WNW           & -                                                                                & 24, 25, 27, 28                                               \\
27 & \begin{tabular}[c]{@{}l@{}}Brühl/Zöllnerstraße/\\ Zöllnerplatz\end{tabular} & NNO           & -                                                                                & 24, 25, 26, 28                                               \\
28 & Zöllnerplatz                                                                & S             & Der verrückte Stuhl                                                              & 24                                                           \\
30 & Brühl/Hermannstraße                                                         & NW            & Urteil des Paris                                                                 & 9                                                           
\end{tabular}
\end{table}



\chapter{Implementierung}
Der Player ist in den grundlegenden Webtechniken HTML, CSS, JacaScript und XML umgesetzt wirden. Da keine serverseitigen Technologien wie etwa PHP oder ähnliche zum Einsatz kommen, ist kein Webserver notwendig, um auf den Player zugreifen zu können. Aus diesem Grund sind die Anforderungen an das System des Endgerätes, auf dem der Player betrachtet werden soll, sehr gering. Es wird lediglich eine funktionierende Internetverbindung, ein Videoausgabegerät, wie zum Beispiel ein Display, und eine Software, die die verwendeten Webtechnologien zu interpretieren weiß, benötigt.

\section{Verwendete Bibliotheken}
Folgende Bibliotheken wurden im Rahmen der Projektimplementierung genutzt:
\begin{itemize}
\item Video.js (Player) \\ Open-source HTML5 Video Player
\item Video.js-Overlay-Plugin (Player Overlays) \\ POI-Overlay
\item Bootstrap (UI Elements) \\ Grid-Layout (siehe Bootstrap-Dokumentation, z.B. col-md-4)
\item jQuery (common lib for JS) \\ Easy access and traversing through DOM Tree
\item XML2JSON \\ Convert XML to JSON-Format \footnote{\url{https://github.com/sparkbuzz/jQuery-xml2json}}
\end{itemize}
\pagebreak
\section{Aufbau der XML-Datei}
Die XML-Datei enthält den Wurzelknoten \glqq interactive\_media\grqq, in dem sich der Knoten \glqq videos\grqq{} befindet. Unter diesem Knoten sind jegliche Videos mit dem Tag \glqq video\grqq{} und einer jeweils einzigartigen Video-ID aufgelistet. Zu jedem Video sind folgende Informationen in der XML-Datei enthalten:
\begin{itemize}
\item Video-ID
\item Dateiname
\item Straße beziehungsweise Straßenkreuzung
\item Blickrichtung nach der Himmelsrichtung
\item Geographische Koordinaten der Position
\item Point-of-Interest mit Bezeichnung, Beschreibung und gegebenenfalls einem Link zur Website (falls vorhanden)
\item Verknüfte Nachbarvideos mit entsprechender Video-ID sowie der Art der Verknüpfung (Richtung)
\end{itemize}

\section{Installation}
Die Installation des Projektes erfolgt über das Klonen des entsprechenden Github-Repositories\footnote{\url{https://github.com/benkindt/video-bruehl}}, dem Überprüfen der funktionierenden Internetverbindung sowie dem anschließenden Öffnen des Players über die Datei \glqq video-bruehl/src/index.html\grqq. Weiterhin muss der Internetbrowser imstande sein, auf lokale Dateien zuzugreifen. Dies erfolgt beispielsweise im Falle das Google Chrome Browsers über den zusätzlichen Startparameter \glqq –allow-file-access-from-files\grqq. Die Videos werden automatisch vom Webserver abgerufen.



\chapter{Zusammenfassung}
Im Rahmen des Projektes \glqq Interactive Media\grqq{} wurde ein Videoprojekt erstellt, welches Benutzern ermöglicht, sich virtuell über den Brühl Boulevard in Chemnitz zu bewegen. Zusätzlich zur Möglichkeit, sich den Brühl anzuschauen ohne wirklich körperlich anwesend zu sein, werden dem Benutzer zusätzliche Informationen zu bestimmten interessanten Punkten zur Verfügung gestellt.\\ \\
Ich stellte sich im Verlauf der Projektarbeit heraus, dass die Implementierung von zusätzlichen Funktionen abseits des Haupvideoplayers sowie der Texteinblendungen sehr kompliziert ist, wie beispielsweise die Bereitstellung einer interaktiven Landkarte. Je mehr Funktionen zur Verfügung gestellt werden sollen, desto mehr Schnittstellen, Bibliotheken und Videoinformationen müssen bereitgestellt werden und miteinander harmonisieren. \\
Weiterhin kristallisierte sich heraus, dass zeitbasierte Annotationen für unseren spezifischen Anwendungsfall nicht sinnvoll sind, da die von uns verwendeten Videos statisch sind.